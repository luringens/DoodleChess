\documentclass{article}
\usepackage[utf8]{inputenc}
\usepackage[left=3.5cm, right=3.5cm]{geometry}
\usepackage[sharp]{easylist}

\setlength\parindent{0pt}
\title{Process and Project Plan}
\author{Group 6}

\begin{document}

\maketitle

\section*{Process and Project Plan}

We will be following an incremental development process. This means we will define a set of features for v0.1 and work towards achieving those. Then we will define a set of features for v0.2, etc, until we have implemented all required features from the product specification. This allows us to tackle the project one bit at the time.\\

To help with this, we will be following an agile method. This means we create a list of tasks to be done, pick which to focus on this sprint, and start working. Next meeting we review what went well and what did not, and adjust our next sprint accordingly. This has several advantages:\\

\begin{easylist}[enumerate]
    # If a task proves more difficult than anticipated, we can further break it up into parts and assign more people to work on it.
    # If anyone finds a type of work unsuitable for them, we can quickly reassign them to a task they can work more efficiently with.
\end{easylist}\mbox{}\\

The scrum metod will require some modifications --- we won't have a product owner involved in the loop, and we won't have daily scrum meetings as a convential scrum team would have as we don't work full-time on the project. Instead, we will twice a week.

\section*{Software development activities}

In addition to producing raw code, we will also require documentation. This can partially be covered by Javadoc for the code, but further documentation is expected to be required. We will also need test suites to ensure correct implementation once we have some basic code ready to test (or at least their interfaces).

\section*{Project management activities}

Project management will be relatively hands-off. The main task will be to schedule meetings and make sure they stay on-topic and focused. Tasks will be made and assigned as a group according to what people feel they are best suited to work on, but it's the ScrumMasters responsibility to ensure that everyone has something to do and to follow up on it next meeting.

\section*{Communcation}

Communication inbetween meetings will mainly be done through Discord, but Facebook Messenger remains an option if you need an immediate respond as everyone has it installed with notifications enabled. 

\section*{Tools}

\subsection*{Version management}

Git has been chosen as our version control system. Any questions regarding the use of this tool can be sent to Stian, who is happy to help with anything from branching guides to merge conflicts.

Our repo structure will be a master branch for versions that are in a good state (v0.1, v0.2, v1.0, etc.), while all other development happens on the develop branch. We should strive to keep each commit on develop compilable --- feature branches can be created for work that will leave the project uncompilable for some time.

\subsection*{Building}

Our build tool will be Maven, as it is a standard tool we were all to some degree familiarized with in Oblig 1.

\subsection*{Testing}

Testing will be done with JUnit tests as that is what we are most comfortable with, ran with Maven.

\subsection*{Issue and task tracking}

We will keep track of issues and tasks, their priority, scope and assignees using GitLab issues for simplicity.

\subsection*{Issue and task tracking}

GUI will be handled by Robin using LibGDX for versatility.

\section*{Meetings}

Meetings will be held every Tuesday 12:15--14:00, and Thursday 14:15--16:00. Each Scrum meeting will have a predefined agenda arranged by ScrumMaster / project lead, and any team member may request topics to be added to the agenda. Defining and allocating tasks will be done as a group, with ScrumMaster taking responsibility to make sure everyone has something to do as well as following up on it next meeting.\\

A generic agenda may look like this:\\

\begin{easylist}[enumerate]
    # Review of progress since last meeting
    # Defining new tasks to be done (if any)
    # Assigning and/or reassigning tasks (if necessary)
\end{easylist}

\end{document}
