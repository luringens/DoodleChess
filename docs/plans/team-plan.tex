\documentclass{article}
\usepackage[utf8]{inputenc}
\usepackage[left=3.5cm, right=3.5cm]{geometry}
\usepackage[sharp]{easylist}

\title{Team Plan}
\author{Group 6}

\begin{document}

\maketitle

\section*{Skills and roles}

\begin{tabular}{l l l}
    Name                   & Specialization \\
    \hline
    Sverre Magnus Engø     & Code \\
    Loc Tri Le             & Code \\
    Benjamin Dyhre Bjønnes & Diagrams \\
    Vegard Itland          & Documentation \\
    Eirik Jørgensen        & Graphics \\
    Robin el Salim         & Test suites \\
    Robin Grundvåg         & Test suites \\
    Stian Soltvedt         & Version control \\
\end{tabular}

\section*{Version control}

A de-facto policy of using the syntax "[Affected file] change description" or "[Topic] content description" in commit message titles has been adopted. Furthermore, a de-facto policy of pushing meeting notes directly to the master branch has taken effect. This policy will be subject to review and a high probability of change in later meetings.

\section{Communication}

We will be using Discord for most forms of communication due to the handy ability to create multiple channels for chat. Facebook Messenger may be used when prompt response is needed.

A Trello project has been established. and will be used to keep track of tasks to perform, and who is assigned to them.

\section*{Risks}

Possible risks, how respond to reduce effect.
Short analysis of risks and mitigation actions.

\subsection*{Feature creep}

We could fall into the trap of overpromising on the number of features added to the application, delaying the project.

We mitigate this by focusing on the basic features required by the assignment in the initial sprints. If members wish to work on additional features, they will have to be done in a later sprint after the minimum required features are already implemented.

\subsection*{Bad time estimations}

We may end up missing deadlines due to underestimating the amount of work required to implement features.

We mitigate this by meeting bi-weekly so we can review progress on tasks. That way we can get quick feedback on whether an estimate is off or not, and take corrective measures.

\end{document}
