\documentclass{article}
\usepackage{booktabs}
\usepackage[utf8]{inputenc}
\usepackage[left=3.5cm, right=3.5cm]{geometry}
\usepackage[sharp]{easylist}

\title{Team Plan}
\author{Group 6}

\begin{document}

\maketitle

\section*{Skills and specializations}

These roles are not final, but serve as a reminder of what people are good at. \\

\begin{tabular}{l l l}
    Name                   & Role \\
    \midrule
    Robin Grundvåg         & GUI \\
    Benjamin Dyhre Bjønnes & Code \\
    Eirik Jørgensen        & Code \\
    Vegard Itland          & Documentation \\
    Loc Tri Le             & Graphics \\
    Robin el Salim         & Test suites \\
    Sverre Magnus Engø     & Test suites \\
    Stian Soltvedt         & Meeting lead, code and Version control \\
\end{tabular}

\section*{Version control}

A de-facto policy of using the syntax ``[Affected file] change description'' or ``[Topic] content description'' in commit message titles has been adopted. Furthermore, a de-facto policy of pushing meeting notes directly to the master branch has taken effect. This policy will be subject to review and a high probability of change in later meetings.

Our repo structure will be a master branch for versions that are in a good state (v0.1, v0.2, v1.0, etc.), while all other development happens on the develop branch. We should strive to keep each commit on develop compilable --- feature branches can be created for work that will leave the project uncompilable for some time.

\section*{Communication}

We will be using Discord for most forms of communication due to the handy ability to create multiple channels for chat. Facebook Messenger may be used when prompt response is needed.

A Trello project has been established and will be used to keep track of tasks to perform, and who is assigned to them.

\section*{Risks}

\subsection*{Feature creep}

We could fall into the trap of overpromising on the number of features added to the application, delaying the project.

We mitigate this by focusing on the basic features required by the assignment in the initial sprints. If members wish to work on additional features, they will have to be done in a later sprint after the minimum required features are already implemented.

\subsection*{Bad time estimations}

We may end up missing deadlines due to underestimating the amount of work required to implement features. We consider this the most likely, and most dangerous risk to our project.

We mitigate this by meeting bi-weekly so we can review progress on tasks. That way we can get quick feedback on whether an estimate is off or not, and take corrective measures.

\subsection*{Integration issues}

As this is our first time cooperating on a larger software project, we may have issues combining all the separate parts individual programmers develop.

We mitigate this by discussing the interfaces we will be developing in advance, so that others can work with those while the implementations are being developed.

\subsection*{Missing key competence}

For some technologies (git, latex) we only have one person who has a lot of experience using it. To mitigate this issue, we have created a \#help channel in Discord where those people are available to help out with the respective technologies.

\end{document}
