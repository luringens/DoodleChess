\documentclass{article}
\usepackage[utf8]{inputenc}
\usepackage[left=3.5cm, right=3.5cm]{geometry}
\usepackage[sharp]{easylist}
    
\title{Retrospective}
\author{Group 6}

\begin{document}

\maketitle

\newpage

\section*{The story / the good}

When we got the oblig there was some confusion regarding the groups due to changes in decision regarding the group selection. We assumed that we would be assigned groups, but it turned out that we ourselves should decide on the groups. This was actually a nice introduction and preparation in working as a group as we had to, as a group, decide on whom we should be grouped with and split in a way that would be both fair and allow people that wanted to be in a group together be in a group. \\

\noindent
When we decided on the grouping were we quite quick to establish a few things:
\\
\begin{easylist}[itemize]

# When to meet
# Establishing what the skill levels of the individual members was
# What role each of us would have in this assignment

\end{easylist}

\noindent \\
In the very beginning, we had to determine meeting timeslots which respected everyone's schedule and would allow all the team members to attend the meetings. We decided on meeting both Tuesdays, in the obligatory group, and Thursdays. It had to be more frequent than only on Tuesdays, and Thursday was a good fit.

\noindent \\
By establishing the skill level early was it easy to assign more general tasks. Since Vegard excelled in LaTeX we decided to make him the person in charge of making sure that the documentation was correctly formatted. Stian excelled in git, so we decided to make him take charge of how to format git commits and have a general overview of the git repo. This way he could easily answer potential questions regarding git. Sverre said that he had lead groups earlier, so we decided to make him the general leader of the group so that he could have a more general overview of the task and make sure that it all comes together in the end. The rest of us took a more general role and mainly helped with the rest of the tasks in the assignment.

\newpage

\section*{The bad}

It wasn't all sunshine and roses, we had some troubles and hardships along the way.
\noindent \\
To name a few:\\

\begin{easylist}[itemize]

# Difficulties in assigning tasks
# Communication issues
# Focus
# Tools

\end{easylist}

\noindent \\
First off, we had some difficulties dividing the assignment into enough subtasks to keep the entire group busy. This resulted in coordination issues and some people might not have been entierly certain of what they were supposed to do. Some tasks were also dependent on others, which meant that some people would sometimes have to wait for others to be able to progress on their own task and might not have had anything to do in that time period. However, since we decided on having such frequent meetings were we usually able to quickly correct and fix potential uncertainties. After moving to a better communication channel, it also became a lot easier to ask questions between meetings, but we still had issues, and a lot of the minutes of meeting was designated to assigning tasks.

\noindent \\
In the beginning we also had some problems with deciding on how we should communicate. At first we decided on Facebook Messenger, but after some time we saw that this was just cumbersome and difficult to keep up with, so we moved over to Discord. After moving to Discord it was a lot easier, due to better notification settings, support for different communication channels and the ability to easier search through messages to find the relevant ones.

\noindent \\
Another problem was keeping focus. If someone had some small issue during the meetings it often happened that the entire group just lost focus since we suddenly would split into subgroups, some who tried to help the one with issues and others who tried to keep a conversation going. This was a problem since we sometimes would bring up important topics during those moments, and later it would turn out that some people in the group never got the message due to the division of attention. This improved over time, and got better once the team lead took a more authorative role and demanded attention when needed.

\noindent \\
The last issue we want to discuss is tools. In the beginning we didn't really focus on using any tools other than the ones specified in the assignment (git/ LaTeX), but some tools were suggested, like Trello. While we werent using Trello in the beginning, we begun using it towards the end. By using it, assigning tasks became easier, and we got a better general overview of which tasks were left to do. Due to the structure of the Trello project board, we also fell into a more "agile" workflow by using it, keeping the tasks which had to be done by next meeting in the "sprint" column in order to separate the essential focal points from the general "backlog" og tasks which would have to be done at some later point. For this assignment we decided that one sprint would be until the next group session so that we could quickly address the tasks which should be done. We think that this workflow worked pretty well and it is something that we are inclined to use going forward.

\newpage
\section*{The future}

In the future, we will try to address the issues described above, while also trying to keep up the things we've done well thus far. When we start programming, we also have some frameworks we want to try using, for instance JUnit for testing and libGDX for UI. We intend to use Maven for building the project, and we will be developing using Java SDK 8. Of course, we'll also use Git for version control, and document our code using JavaDoc.

\end{document}
