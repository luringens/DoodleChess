\documentclass{article}
\usepackage[utf8]{inputenc}
\usepackage[left=3.5cm, right=3.5cm]{geometry}

\title{Minutes of meeting \#4}
\author{Group 6}
\date{Thursday, 2018-02-15 14:15-16:00}

\begin{document}
\maketitle

\section{Attendance}
\begin{tabular}{l l}
    Benjamin Dyhre Bjønnes     & present \\
    Robin el Salim             & present \\
    Sverre Magnus Engø         & present \\
    Robin Grundvåg             & present \\
    Vegard Itland              & present \\
    Eirik Jørgensen            & present \\
    Loc Tri Le                 & present \\
    Stian Soltvedt             & present
\end{tabular}

\section{Pre-meeting catchup}

\paragraph{Last meeting recap}

We went through what had happened during the last meeting for members who had to leave early.

\paragraph{Group chat platform change}

After lobbying from some members, the group switched from Facebook messenger to Discord for group communication outside of meetings, due to its superior notification control, multiple channels, flexibility and general advantages over Facebook Messenger. This was a de-facto policy change that took place outside of physical meetings.

\paragraph{Role assignment}

Robin E has taken on the role as the official Chief Excecutive Bullshitter (CEB) of this project.

\paragraph{Process and project discussion}

We went through the description of the group's process and project plan, and discussed some topics which needed the approval of the whole group. The points discussed were:

\subparagraph{Documentation}

We want to use JavaDoc comments to automatically generate the documentation, due to the simplicity and the fact that this is, to the point of standardization, a widespread way of documenting (Java) software projects.

\subparagraph{Tests}

We intend to write as many tests as possible, in order to ensure that everything is working properly. It was suggested that even before we have working code, we should define the interface of the classes we intend to make so that a tester can in parallel start to develop tests according to how we expect these classes to behave.

\subparagraph{Project management activities}

It was affirmed that the project management activities were to be kept on a low level, mainly restricted to organizing meetings and making sure that everyone has somehting to do.

\subparagraph{Communication}

Discord was officially adopted as the day-to-day mode of communication, with Facebook Messenger reserved for critical (important/urgent) messages due to people more easily receiving notifications on that platform.

\subparagraph{Trello}

We recognize that the team as a whole lacks experience with Trello, but we also see that the functionality it provides may prove useful in organizing and dividing tasks among team members. We will try to incorporate it into our workflow. In all likelihood, we will reserve adding items to the Trello project for meetings, and work on them in-between meetings.

\paragraph{Taco Tuesday}

A team member suggested that we organize Taco Tuesdays as a sort of an official group lunch during meetings, but nobody seemed very willing to make any efforts towards purchasing the ingredients or making the taco.

\section{Main topic: Group coordination}

Aside from topics brought up in the catchup segment, there was not a lot to discuss in the group as a whole. The main part of the meeting went to resolving questions and problems individual team members had, clearing up parts of the process, and otherwise discussing where we are and where we're going. We used Trello to assign tasks to members for next meeting.

It was decided that we cannot begin on neither the retrospective task nor the presentation task yet, since they almost by definition come at the end of the entire project.

The plan is that the deliverables for all previous tasks will be done by next meeting.

\paragraph{Usage of tools}

A few things were cleared up about how to use LaTeX, and people have started pushing deliverables to the team repository. Team members who are new to LaTeX are advised to use Visual Studio Code with an extension which automatically compiles the LaTeX document to a PDF upon saving the file, because of the simplicity compared to introducing a command line tool with a steeper learning curve.

Any and all questions about Git can be asked to @Stian in \#help in Discord.

\paragraph{Usage of frameworks}

\subparagraph{Testing framework}

We'll be using JUnit for testing due to it being the standard Java testing framework which most of the members have the most experience in.

\subparagraph{GUI framework}

We have decided to use JavaFX for this project, due to it being a technology most of us have at least a working level experience with from INF101, as well as having a lot of effects functionality which will be useful when developing a game with graphics, as contrasted with a tool built purposely for accomplishing a task. Most of the members seemed somewhat indifferent to what framework we were going to use, except for strong opinions that we should use something which is part of the standard library, on account of it being way easier than making all the members install a third-party framework.

\section{Meeting review}

\paragraph{What worked?}

Due to its lesser intensity, compared to previous meetings, this meeting was a lot easier to take notes about than usual. It seems like we managed to clear up some practicalities regarding the technical aspects of the collaboration, such as Git, LaTeX, etc.

\paragraph{What didn't work?}

There were a lot of silent stretches throughout this meeting, and we didn't really have a lot of an agenda to go through. Before we can start on the next subtasks, we need to complete the tasks that come before.

\section{Conclusion}

We've cleared up where we're at, reaffirmed assigned tasks, made decisions on usage of tools and frameworks, and answered questions by individual team members.

\paragraph{Next meeting}

The next meeting will be held in Høyteknologisenteret, group room 209N1, on Tuesday 2018-02-20 at 12:15-14:00

\end{document}
